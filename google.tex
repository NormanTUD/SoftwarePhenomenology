\section[\google]{Then now, why is \google\ so successful?}

\frame{
	\begin{itemize}
		\item Because \google s founder's the world phenomenological, probably without realizing it.
		\item They wanted to solve a real problem, finding information, on the real internet as quickly and with as little hassles as possible.
		\item It's pure phenomenological Software testing with the goal of solving a problem in as little steps as possible.
	\end{itemize}
}

\frame{
	\begin{itemize}
		\item They see a problem. You are outside, away, and need a timer for something. You always have smartphone with you, would be great if it
			had a timer. So they add one.
		\item Tired of typing with cold fingers, very slowly on a screen? Develop speech recognition.
		\item And now you can add those. You can teach the speech recognition to set a timer for you.
	\end{itemize}
}

\frame{
	\begin{itemize}
		\item Maybe you want to go out sometimes, go for a fully new place. It would be great to have a digital map and a navigation website.
		\item Yeah, use \google\ Maps. You can even use speech recognition for the map, and output your speech so that you don't have to read.
	\end{itemize}
}

\frame{
	\begin{itemize}
		\item When you drive somewhere, the goal is not the interaction with the navigational system, but it's reaching the actual goal. The navigation is just
			a thing that you need to reach a goal convienently and it should annoy you as little as possible.  
		\item What they do: they continue the road laid out in the CPU-Design-Part. They abstract the concrete details the most away from you as possible
			because most people don't care about those details. They just want a quick result.
	\end{itemize}
}


