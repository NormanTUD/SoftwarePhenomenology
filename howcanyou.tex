\section[Summary]{Summary}

% There are different modes of being. One is the flow, another one is distress. We dislike distress and like "the flow". In "the flow" we concentrate fully on what we do, and achieve exactly what we want to achieve. The flow is one of the most basic things that tell us that something is good, and distress, which destroys the flow, is one of the most basic things that tell us something is bad.

\frame{
	Good software has these 2 properties:
	\begin{itemize}
		\Huge 
		\item it is as abstract as possible,
		\item it \textbf{never} breaks the flow!
	\end{itemize}
}

\frame{
	Those are the steps from a phenomenological standpoint that you have to follow to create good software:
	\begin{itemize}
		\item Look what you yourself (and others with other mindsets) \textit{really} do and fix the program until it matches the
			way people who will use the software think
		\item Try seeing what `annoying' means by completing the userinyerface.com task to `register'. You will learn a lot by doing this.
	\end{itemize}
}
