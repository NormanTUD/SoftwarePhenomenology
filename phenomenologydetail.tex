\section[Phenomenology]{What is \frqq Phenomenology\flqq?}

\fullpageimage{black}{heidegger-crop.jpg}{heideggerbild}
\note{
	\begin{itemize}
		\item Heidegger was the first philosopher to criticize the european philosophy at it's fundamental level.
		\item He said, from Plato on, philosophers have lost track of `real life', focussing on abstractions instead.
		\item Using abstractions is easier of course, but an abstraction can never replace real life.
		\item He called it `Seinsvergessenheit', the forgetting of being itself.
	\end{itemize}
}

\fullpageimage{black}{heidegger-crop.jpg}{heideggerbild}
\note{
	\begin{itemize}
		\item You can think about riding a bike, or ride a bike. They're not the same at all.
		\item This is called ontic-ontological difference. Ontic refers to `what we really do`, ontological is `thinking about what we really do'.
		\item The task of philosophy is, according to Heidegger, to get the the focus to the real life instead of just thoughts about real life.
	\end{itemize}
}

\fullpageimage{black}{heidegger-crop.jpg}{heideggerbild}
\note{
	\begin{itemize}
		\item This is what we have to keep in mind for good software, too. 
		\item Move away from what you think users want and are to what users really want and are.
		\item This is not easy, since most of the time they don't know it themselves.
		\item But it can be achieved by observation of yourself, since you are also `just' a user of your own software.
		\item And it can be achieved by outside user's with different mindsets and goals via communication.
		\item Schmitz, a successor of Heidegger, has really started this focus on real life, in my opinion.
	\end{itemize}
}

\fullpageimage{black}{schmitz.jpg}{schmitzbildwiki}
\note{
	\begin{itemize}
		\item This is Hermann Schmitz.
		\item He's a comtemporary german philosopher who invented `the new Phenomenology'
		\item Sadly, he passed away on the 5. of may 2021 at the age of 92.
		\item His ideas in short terms were:
		\begin{itemize}
			\item We are not merely our rationality, nor are we merely our body. 
			\item Our rationality is that which thinks most often in words inside our head.
			\item Our body is that which a doctor palpates when he examines us.
			\item What's missing is our \textit{subjective body}. If a loved one touched you exactly the same
				as a doctor may touch you, it may feel very different.
			\item The subjective body is our experience of our body \textit{and} our mind.
		\end{itemize}
	\end{itemize}
}

\fullpageimage{black}{schmitz.jpg}{schmitzbildwiki}
\note{
	\begin{itemize}
		\item Do you know phantom pain? 
		\item If someone loses an arm due to an accident, it would be logical to think that he does not feel it anymore.
		\item But this is not the case. People who lost limbs often report still feeling the pain in them.
		\item This is because though they lost a part of their physical body, it is still part of their \textit{subjective body}.
		\item In this \textit{subjective body} we experience all our emotions and atmospheres.
		\item All of live is basically just a row of following situations in certain spaces, submerged in atmospheres.
		\item This is important to notice. You cannot not be in a situation. You cannot not be in an atmosphere.
			They sorround you all the time. 
	\end{itemize}
}


\fullpageimage{black}{schmitz.jpg}{schmitzbildwiki}
\note{
    \begin{itemize}
        \item Atmospheres and situations, not physical matter, primarily constitute reality as we experience it
        \item The new idea of Phenomenology is to treat experiences as real. The pain you may experience is as real as the chair your sitting 
            on
        \item They are in different categories, since you have access to that chair, but not my experiences, but they are nonetheless
            the realest reality I can get
        \item If you are in severe pain, the pain is much more real to you than a mathematical proof
        \item Good programs create a good atmosphere. Programs that create bad atmospheres (by, like, crashing, or acting unexpected)
            are bad programs.
    \end{itemize}
}



% TODO!!!! Im Video Sound einbinden. creepy.mp3!!!
\fullpageimage{black}{creepy.jpg}{creepyimage}
\note{
	\begin{itemize}
		\item !!! Add creepy music here !!!
		\item Imagine walking through this abandoned insane asylum in the night without a flashlight, hearing weird sounds
		\item Do you feel that constricting feeling in your chest-area? This is your \textit{subjective body}s reaction to 
			that creepy, weird area.
		\item The german Word `Angst' is etymologically derived from the same word as `Enge', which means `narrowness',
			because it's connected to this constricted feeling.
	\end{itemize}
}

\fullpageimage{white}{euphoria.jpg}{wikieuphoria}
\note{
	\begin{itemize}
		\item Or have you ever felt like this? Liberated, euphoric, happy to be alive?
		\item The feeling in your chest-area might be liberating and widening.
		\item According to Schmitz, these are the most general ways of subjective body states:
			between absolutely constricted, and absolutely open
		\item You need both in life, like the expansion of the Leib when breathing in in swelling of the fealing chest-area, that would be
			unbearable if it wouldn't be exhausted just right after, resulting in in wave of experiences of widening and narrowing Leib
		\item Usual situations are a complex mixture of states between those extremes in different what Schmitz calls 'Leibliche Region', like the lost arm with phantom pain, ``that are in the area, but not neccessarily in the borders of the body''.
		\item Whether you feel fear, or you are happy or liberated, it is always a feeling that affects the Leib
	\end{itemize}
}


\fullpageimage{white}{euphoria.jpg}{wikieuphoria}
\note{
	\begin{itemize}
		\item These are the most basic extremes of life. This is the best that 4 billion years of evolution
			could produce to tell us about our sourroundings.
		\item We should start to use it properly!
		\item That is, have our inner focus properly adjusted to experiences.
		\item If you use your software, and something annoys you, which you can fix, \textit{fix it}! (This is a generally good tip for life)
		\item But you have to really use the software. And see how other people use your software. They'll often don't see things that 
			seem `obvious' to you because your so involved with your code.
	\end{itemize}
}

\fullpageimage{white}{euphoria.jpg}{wikieuphoria}
\note{
	\begin{itemize}
		\item They also see things that are obvious to them, but not to you, since you are not thinking like them, because you are not them.
		\item Heidegger calls this ``Jemeinigkeit'', the \textit{Dasein}, i.\,e. \textit{you} as a human being, is always experienced from 
			the subjective perspective that is always \textit{mine} (from the person observing)
	\end{itemize}
}


\fullpageimage{white}{rtfm.jpg}{wikirtfm}
\note{
	\begin{itemize}
		\item In theory, users have an infallible memory system. And they read manuals before using the program. They know exactly what they want 
			and how to achieve it the most efficient way possible.
		\item But in reality they are never like that. 
		\item Do you read a 400 page manual before using a `simple' program? I do not. Nobody wants to read a manual to
			use a webbrowser. It's just a tool that needs to do it's job. And of course real humans always forget and barely ever know
			what they want exactly.
		\item It's not like when you play music on Youtube, you want to think about how the https-protocol is implemented. You only care about
			the music.
	\end{itemize}
}

\fullpageimage{white}{rtfm.jpg}{wikirtfm}
\note{
	\begin{itemize}
		\item Thinking is a very complex task and requires a lot of effort that the people using software would rather like to spend
			on the problem they try to solve \textit{with} the computer
		\item I believe we developed consciousness evolutionary to have the ability to focus on problems that we cannot solve by intuition alone.
		\item Remember when first learning to drive a bike or a car? You had to do everything very consciously, until it went `down in
			your nervous system' so that you do the things correctly without thinking about them. When consciousness fulfils it's
			duty of learning, we can disable it and drive on auto-mode mostly.
		\item The task of consciousnes is, as Zen buddhism puts it, to get rid of consciousness.
		\item This is what he have to achieve with our software, so that people will use it without having a learning curve. If your
			users always use the program wrongly, they think different about the problems than you, and your task is to bridge that
			gap by making it work the way people think.
	\end{itemize}
}

\fullpageimage{white}{rtfm.jpg}{wikirtfm}
\note{
	\begin{itemize}
		\item Real humans have a very limited ability for mental workload. They can only keep some very few things at the same time in their head, and usually
			the computer is one of those that they have to have in their head, but don't want it there, because they don't have the goal of
			using the navigation software, but of reaching the goal \textit{with the navigation software}.

		\item They want a result on the level of analysis they look at right now
		\item If you drive a car and it breaks down, you have to change your level of analysis. From \textit{this is an object that gets me from $A$ to $B$},
			you must suddenly switch to \textit{this object has an engine with like a billion parts and complex circuits and all that stuff, what do now?}.
		\item A good car is one where you don't have to switch the level of analysis. And this is also the trademark of great software.
	\end{itemize}
}


\fullpageimage{white}{hegel.jpg}{wikirtfm}
\nocite{hegeldeutschlandfunk}
\note{
	\begin{itemize}
		\item Hegel wrote an article about abstractness and concreteness (\cite{hegelabstrakt}). He said, usually people think the abstract thing is the hard one, like abstract mathematics, and the concrete is the easy one. He argues that often it's the opposite.
		\item For example, checking my average school marks vs. the ones' of another one is easy. But looking at whether the person is right in that job, that's hard.
		\item Abstract means, we exclude. It is etymologically derived from abstrahere in latin, which means to 'cut away'. Abstract means we can concentrate on only one or at least very few factors.
		\item Abstraction is great, because then we can see parts of the future and develop towards them, even though they are not here yet. This gives us the vision to do things. But it comes with the price that we all-to-often forget the reality, fall for Heideggers' `Seinsvergessenheit', and confuse our image of the world with the real world.
	\end{itemize}
}

\fullpageimage{white}{hegel.jpg}{wikirtfm}
\note{
	\begin{itemize}
		\item In software, we want exactly \so{this} abstraction.
		\item Good software cuts away everything from you that you don't care about while solving the problem
	\end{itemize}
}

\fullpageimage{white}{tunnelradio.jpg}{wikirtfm}
\note{
	\begin{itemize}
		\item Do you know the feeling of flow? When you do something and everything works as expected and helps you to fulfil your goal?
		\item Sometimes, when driving through tunnels, you see signs that the radio will not work. 
		\item You want to drive a car and listen to radio. Of course you know why it has problems. You kind of know how the radio network works.
		\item But This is an interruption nonetheless. Where the flow is interrupted.
		\item NEVER EVER BREAK THE FLOW
		\item Flow-breakers are needing to think about the implementation of something that you just want to use, like the navigation software,
			or they even outright stop working, like the radio.
	\end{itemize}
}


\fullpageimage{white}{tunnelradio.jpg}{wikirtfm}
\note{
	\begin{itemize}
		\item Every time something breaks the flow, fix it.
		\item We cannot expect all our users (and ourselves) to become full ``zen-monk-flowy''. So we have to take care about all the situations that break the 
			flow by fixing them.
		\item Think in user-perspective. The user does want to solve a problem, not first create a bunch of other problems for himself.
		\item While there are flow-breakers, fix them. Fix the most severe ones first (like not being able to start the software
			without a bluescreen).
		\item The second biggest flow-breaker has become the most important show-stopper now, which you also fix. The software is 'done'
			if neither you nor the end users (or a test group of normal users) find some other flow-breakers.
		\item (But in reality, it's never done.)
	\end{itemize}
}


\fullpageimage{white}{tunnelradio.jpg}{wikirtfm}
\note{
	\begin{itemize}
		\item Eat your own dogfood! Actually use your software. And look at how end-users will use it. They will think about it
			differently, because they don't have the knowledge about it you do.
	\end{itemize}
}

\fullpageimage{white}{speechrecognition.jpg}{speechrecognitionimage}
\note{
	\begin{itemize}
		\item Things like speech recognition does all of this. The user does not have to think about a machine as machine,
			but can only say what they want
		\item With the rise of language models like chatGPT, this development is much further driven than ever imagined in scifi-movies already.
		\item This abstracts away completely even from using the `normal' interface
		\item This is Google's trick. You don't need to care about implementation details at all.
		\item The web usability tester Steve Krug said about this: `Don't make me think!'
		\item This is the secret.
		\item Good software is the software that doesn't make you think
	\end{itemize}
}


