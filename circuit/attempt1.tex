\documentclass{standalone}

% These are only some keywords for the autocompletion-feature of many editors: section, subsection, subsubsection, paragraph,
% includegraphics, width, linewidth, linespread, figure, wrapfigure, caption, label, footnote, equation, input, cite, citetitle,
% citeauthor, footfullcite, tableofcontents, printbibliography, clearpage, frq, frqq, flq, flqq, grq, grqq, glq, glqq, textit,
% texttt, mathrm, dots, pmatrix, centering, phantom, minipage, ensuremath, hfill, vfill, 

\newcommand{\centeredquote}[2]{
	\hbadness=5000
	\vspace{-1em}
	\begin{flushright}
		\item\frqq\textsl{#1}\flqq\ 
	\end{flushright}
	\nopagebreak
	\hfill ---\,\textsc{#2}\newline
	\vspace{-1em}
}

\newcommand{\centeredquoteunknownsource}[1]{
	\hbadness=5000
	\vspace{-1em}
	\begin{quotation}
		\begin{flushright}
			\item\frqq\textsl{#1}\flqq\ 
		\end{flushright}
	\end{quotation}
	\vspace{-1em}
}
\usepackage[utf8]{inputenc}
\usepackage[T1]{fontenc}
\usepackage[sc,osf]{mathpazo}
\usepackage{fourier}
\usepackage{soulutf8}
\usepackage{graphicx}
\usepackage{amsmath}
\usepackage{amssymb}
\usepackage[ngerman]{babel}
\usepackage{circuitikz}

\emergencystretch2em


\begin{document}

\begin{example}[gobble=0]
\begin{tikzpicture}
  \draw (0,6.8) node [left] {\(+\)} -- (9,6.8);
  \draw (0,0) node [left] {\(-\)} -- (9,0);
  \draw (4.5,0) to[short, *-] (4.5,0) node [ground] {};

  \draw (7.4,2.5) to[short,*-] (7.5,2.5) to[lamp] (9,2.5)
    node[ground] {};

  \draw (2.5,5.8) node[arbeits relais] (a1) {};
  \draw (2.5,4) node[arbeits relais] (a2) {};
  \draw (2.4,6.8) to[short,*-] (a1.anschluss);
  \draw (a1.ausgabe) -- (a2.anschluss);

  \draw (2.5,1) node[ruhe relais] (r1) {};
  \draw (a2.ausgabe) -- (r1.anschluss);
  \draw (r1.ausgabe) to[short,-*] (2.4,0);
  \draw (5,1) node[ruhe relais] (r2) {};
  \draw (r2.ausgabe) to[short,-*] (4.9,0);

  \draw (7.5,1) node[arbeits relais] (a3) {};
  \draw (7.5,4) node[ruhe relais] (r3) {};
  \draw (a3.anschluss) -- (r3.ausgabe);
  \draw (a3.ausgabe) to[short,-*] (7.4,0);
  \draw (r3.anschluss) to[short,-*] (7.4,6.8);

  \draw (2.4,2.5) to[short,*-*] (4.9,2.5) -| (a3.eingabe);
  \draw (r2.anschluss) |- (r3.eingabe);

  \draw (0,4.7) node [left] {A} to[short,-*] (0.2,4.7) --
    (a2.eingabe);
  \draw (0.2,4.7) |- (r1.eingabe);

  \draw (0,2.1) node [left] {B} to[short,-*] (0.4,2.1) -|
    (r2.eingabe);
  \draw (0.4,2.1) |- (a1.eingabe);
\end{tikzpicture}
\end{example}

\end{document}
